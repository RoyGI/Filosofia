\documentclass[12pt,letterpaper, a4paper ]{article}

\usepackage[spanish]{babel}
\usepackage{amsmath}
\usepackage{amsfonts}
\usepackage{amssymb}
\usepackage{makeidx}
\usepackage{graphicx}
\usepackage{kpfonts}
\usepackage[left=2cm, right=2cm, top=2cm, bottom=2cm]{geometry}

\title{\Huge{La existencia de Dios}}
\author{Roy Gonzalez}

\begin{document}
\maketitle

\begin{abstract}
El siguiente articulo tiene como finalidad llegar a la demostración de dios por medio del análisis de la razón. Se presentara una revolucionaria visión que entra en los albores de la metafísica.
\end{abstract}

\section{La existencia de Dios}
Este problema lleva a pensar que el problema no tiene solución, ya que es una entidad metafísica y no se puede llegar a un consenso; En este articulo se tratara de llegar a una demostración basada en 
un convencionalismo, en ese caso, la existencia de Dios quedaría supeditada a una democracia y para que sea universal todos los habitantes del planeta tendrían que votar la existencia, es decir seria una decisión unánime.
En este momento esta expresión se volvería verdadera y nadie podría objetar lo contrario.
Parece que estoy atrapado en una contradicción por un lado hablo sobre la existencia de dios y por el otro lado hablo de la imposibilidad de la demostración desde una perspectiva democrática; sin embargo, sin ser
dialéctico voy a llegar a la consolidación de esta contradicción.


\subsection{El Problema de la metafísica}
 Que es la metafísica? una pregunta muy profunda que vive en el mismo abismo de la metafísica. depende a quien se le pregunte tendrá una respuesta propia de lo que es metafísica. A menos, que exista un grupo o una secta que
 crean en una especifica interpretación. Para el autor de este articulo la metafísica es todo excepto las matemáticas. si todo, las ciencias, la física, la medicina.
 
 Por que todo es metafísica? a diferencia de las matemáticas que son conocimientos propios de la razón, la metafísica son conocimientos basados en la creencia. He aquí lo que es la metafísica. un conjunto de creencias
 que poseen los humanos, este es un hecho innegable ya que la ciencia y las otras áreas del pensar se basa en creencias.
 
 Ahora bien, sera la metafísica una única metafísica, o existirían las metafísicas? la respuesta es definitiva, existen las metafísicas, hay tipos de metafísicas, como el fisicalismo que es una doctrina de explicar todo
 basado en hechos físicos, también esta el sociologismo de explicarlo todo en relación de los humanos (así o mas egocéntricos somos), esta el experimentalismo doctrina que se basa en la experimentación como el caso de las ciencias medicas.
 
 Basado en que las ciencias son creencias, esta creencias son sostenidas por una cultura o por un grupo de individuos que toman estas creencias como hechos, o por lo menos les permite dar un ordenamiento al mundo que los rodea; esta posición es la idea de "" que las ciencias son organizaciones o sectas que toman como verdaderas ciertas teorías.
 
 Resumiendo las metafísicas son las bases de las ciencias, ya que están tienen ciertos principios que se basan en creencias. La teoría que se plantea en este articulo se basa demostrar la existencia de dios a través de la metafísica. Pero no es cualquier metafísica, lo voy a demostrar desde el fisicalismo.
 

\subsection{El problema de la entropía}

Para poder empezar la demostración desde el fisicalismo me gustaría introducir el concepto de entropía, ya que todos los procesos de la naturaleza se ven afectado por este concepto, es decir es una ley universal de la naturaleza.
La entropía afecta los procesos como la información, la transferencia de calor, el expansionismo del universo etc.

La entropía se define como un modelo de desorden. en palabras mas formales, es una razón de cambio de la transición de un estado a otro. Ejemplo, supongase que tienes un baso de vidrio frágil en la mano y lo suelta a una distancia que el baso se rompe en varios pedazos,
los pedazos de vidrios se distribuirían de una forma aleatoria sobre el piso. Esto es que el estado del baso paso de ser baso a un conjunto de pedazos de vidrios esparcidos por el piso de manera aleatoria.

Ahora bien supongase que se toman los pedazos de vidrios y se lanzan al piso
no sucede ni sucederá nunca que a través de los fragmentos de vidrio arrojados al piso se forme un baso de vidrio. al contrario aumentaría la entropía se crearían nuevos fragmentos de vidrio y se incrementaría la perdida de información, es decir el baso deja de ser reconocible.

Entonces si existiera un dios que sea explicado desde el fisicalismo este dios debería estar sujeto a la entropia, lo cual se va a demostrar que dios cumple esta condicion.

\subsection{El problema de la vida}

En esta sección quiero demostrar que la vida es un sistema caótico; este es uno de los axiomas principales de mi teoría por que demostraría la intervención de dios en el mundo. 

La vida es caótica porque podria existir una maquina, una computadora que pueda tomar las entradas de la realidad, hacer cálculos y simulaciones para predecir el mejor futuro, es decir existiría un futuro optimizado. sin embargo tal hazaña requeriría  un esfuerzo tan grande que seria imposible llevar a cabo.

El humano podría sufrir eventos que cambien su vida diaria, como el hecho de que los hombres han perdido el 50 porcentaje de producción de testosterona en los últimos 50 años, o cambios sociales que cambien la dinámica humana. lo que llega a pensar que la vida es un sistema dinámico, caótico, por que el futuro 
depende de variables que si se modifican aunque sea un poco el sistema se vuelve impredecible. Este comportamiento de impredisibilidad se puede ver en las bolsas de valores, el cual no existe ningún sistema que pueda predecir los valores de los mercados.

En resumen la vida es caótica por lo que no puede existir un programa o algoritmo que prediga el futuro. la vida también esta sujeto a las leyes de la entropía.

\subsection{El problema de la conciencia}
En el problema de la conciencia se tienen muchas posiciones, sin embargo, no se van a ver estas posiciones si no que se va a partir de una posición complemente nueva.

La conciencia es una función matemática que aparece de la sumatoria de todas las estados posibles microscópicos a los macroscopicos. es decir que la conciencia va a emerger de estados pequeños a estados mayores. 

Ya que vamos a empezar de microestados se podría decir que la conciencia es un estado cuántico de la naturaleza que se alojan en las neuronas, es decir las neuronas tienen conciencia a nivel atómico.

No se malinterprete que la conciencia tiene propiedades de inteligencias, por que hay conciencias de conciencias. hay conciencias básicas como la conciencia de los animales a otras mas inteligentes como las conciencias humanas.

Entonces tener conciencia no es mas que la conciencia de las conciencias y así sucesivamente hasta un tope (no voy a decir infinito, por que tendría problemas en el fisicalismo), las conciencias formarían un estructura fractal;
eso es la conciencia: serial un fractal de fractatales de conciencia, esta es una definición recursiva de la conciencia.

\subsection{El humano como maquina}

La conciencia emerge de las conciencias mas básicas, el humano en si es una maquina de maquinas, asi, como lo es el humano, es la conciencia: una conciencia de conciencias.

El humano es una maquina de maquinas, por que, a nivel celular hay pequeñas que copian el código del ADN y lo reparan, como también las maquinas para procesar proteínas, sin embargo, estas maquinas que conforman al ser humano no tienen presencia en la conciencia. Son maquinas que constituyen una maquina superior como lo son los humanos. así mismo es la como la conciencia emerge.

\subsection{El cerebro de Boltzmann}

Se suele hacer referencia a estas entidades en el contexto de "la Paradoja del Cerebro de Boltzmann".

El concepto proviene de la necesidad de explicar por qué observamos tal grado de orden en el Universo. La segunda ley de la termodinámica afirma que la entropía en el Universo siempre se incrementará. Podríamos pensar, pues, que el estado más probable del Universo es uno de alta entropía, casi uniforme y desordenado. Así que, ¿por qué la entropía que observamos es tan baja?

Boltzmann propuso que nosotros y el Universo de baja entropía que observamos somos en realidad una fluctuación aleatoria en un Universo de mayor entropía. Incluso en un estado cercano al equilibro, existirán fluctuaciones estocásticas en el nivel de entropía. Las fluctuaciones más comunes serían relativamente pequeñas, únicamente dando lugar a pequeñas cantidades de orden, mientras que mayores fluctuaciones y sus mayores niveles de organización asociados serían relativamente más raros. Enormes fluctuaciones serían altísimamente improbables, pero esto puede ser explicado por el enorme tamaño del Universo y por la idea de que si somos resultado de una fluctuación, hay un proceso de selección implícito: Observamos este universo tan improbable porque tal improbabilidad es necesaria para que estemos aquí.

Todo ello conduce al concepto del Cerebro de Boltzmann: si nuestro actual nivel de organización, dadas muchas entidades conscientes de sí mismas, es el resultado de una fluctuación aleatoria, éste sería mucho más improbable que un nivel de organización que sólo es capaz de crear una entidad consciente de sí misma. Para cada Universo con el nivel de organización que vemos en éste, debería haber una cantidad enorme de solitarios cerebros de Boltzmann vagando en entornos desorganizados. Esto, pues, refuta el argumento previo: la organización que veo es tremendamente mayor de la que es requerida para explicar mi consciencia, y por lo tanto es altamente improbable que yo sea el resultado de una fluctuación estocástica.

La paradoja, en resumen, consiste en que es más probable que un cerebro se forme aleatoriamente desde el caos con recuerdos falsos sobre su vida, a que el Universo que nos rodee tenga billones de cerebros conscientes de sí mismos.

Con esto se demuestra de una existencia de un cerebro superior que nos podría estar imaginando. El axioma presentado anteriormente establece que la vida es caotica, es porte de un proceso del azar. la idea de una conciencia que emerge de recuerdos falso o aleatorias tiene todo el sentido del porque experimentamos la vida o la realidad de forma aleatoria. No es que sean recuerdos falsos, son recuerdos aleatorios.


\subsection{La demostración de la conciencia}

El tiempo es una onda que se propaga por el espacio, no hay un presente, lo que existe es una secuencia de información la cual se le podrían partir en secciones a cada sección se les podría llamar como futuro, presente y pasado.

Si nos ponemos en cualquier posición en el espacio podríamos ver el presente o el futuro, lo cual llamar al tiempo como presente, pasado y futuro no tendría ningún significado, ya que depende desde el punto de vista o de referencia de un espectador.
Esto seria simplemente información que se podría clasificar en estas secciones.


Ahora bien a que es lo que llamamos presente, todos experimentamos el sentido de estar presentes, pero que es estar presente?, este es otro hecho innegable de la existencia de la conciencia, la conciencia es la que nos da
la forma para entender el sentimiento de presencia. El presente no es mas que un sentimiento existente en la conciencia.

El tiempo es una onda ya que los eventos que suceden se dan en un lugar y un tiempo especifico que viajan a la velocidad de la Luz, lo cual se podría entender que el tiempo es una onda, no hay que confundir el espacio-tiempo como una onda
ya que la estructura del espacio-tiempo es eso una estructura que posibilita que las cosas se den en un espacio y un tiempo, es decir, el espacio-tiempo es un campo, en el cual los hechos suceden.




\subsection{La inmaterialidad de la realidad}

En esta sección se va a demostrar la que la realidad no es mas que una ilusión holografíca; que algo sea un holograma no quiere decir que lo proyectado es un reflejo de algo real, así, como en un proyector holografico podemos proyectar una secuencia de la vida real, tambien podemos proyectar una animacion, lo cual lleva a pensar
que lo proyectado es que un holograma podría ser real o no. 

La idea de un holograma viene de la teoría de cuerdas que establece ciertos problemas a nivel del entendimiento de los agujeros negros, que no entraremos en detalle en este articulo.

Sin embargo la teoría de cuerdas presupone que el universo esta creado de pequeñas entidades denominadas cuerdas, forma un campo en donde la energía interactua con la materia a partir de ciertas fluctuaciones en el campo (del ingles field, como el campo electromagnético que no tiene una existencia material) de las cuerdas.

Supongamos que ese campo este lleno de estas cuerdas, estas cuerdas tienen que tener algún tipo de materia o algún tipo de sustancia que permita la creación de materia. si tenemos un campo como este, nos tendríamos que preguntar como interacciona las cuerdas con la materia. Cuando movemos una mano, nos estamos moviendo en el campo de las cuerdas

Sin embargo, no parece ver ninguna limitación de este movimiento ya que aunque se mueva por el espacio-tiempo y sobre el campo de las cuerdas, es decir: las cuerdas no se ven afectadas por los movimiento. Esto implica que la materia no inter-actua con las cuerdas, lo cual es una contradicción. Esto implica que la unica forma de que esto suceda
es que la realidad no sea mas que un mero un holograma.



\subsection{Sobre el universo}
Lo que se postula en este articulo determina que somos una realidad al azar o de ideas falsas; ademas se promulga que la vida es caótico, e aquí los principales axiomas de mi sistema, de ellos se podrían extraer ciertas contradicciones como el hecho, si la vida es caótica, también lo podría ser el universo ya que son hechos generados por el azar. Sin embargo, todas las contradicciones se podrían resolver por el concepto de la entropía, ya que el universo se mueve al azar, pero este tiene un cierto orden, tiene una entropía baja, es decir las cosas suceden aleatoriamente pero con una entropía baja, asi mismo la realidad o la vida, al estar formados por el hechos al azar no determina que haya una entropia alta, es decir la entropía de la realidad es baja. Es por esto que podemos experimentar un cierto orden cronológico.


\subsection{Acerca de la conciencia y el universo}
En la actualidad hay una disputa si existe un universo paralelo, o si existen múltiples universos, sin embargo sobre los universos paralelos se creen que estos son originados por la idea que una partícula puede estar dos veces en un mismo lugar, esto quiere decir que hay estados en las partícula  que tienen la misma probabilidad de darse siendo la misma partícula, si este efecto se magnifica, podríamos estar en dos universos paralelos, sin embargo experimentamos una sola conciencia, una realidad.

Solo podemos experimentar una realidad porque el hecho que existe una baja entropía, es decir la realidad tiene un cierto orden, aunque las probabilidades de que un estado de la partícula se de en dos lugares distintos en el mismo tiempo, solo uno se podría dar en la conciencia. Es decir que la conciencia es superior a la materia y al universo, por que ante la multiplicad de universos, solo experimentamos una sola conciencia. 

El universo podría tener formas de ramificarse, pero de todas ellas solo experimentamos una sola realidad una sola conciencia, lo cual nos lleva a creer que aunque el universo se ramifique, las probabilidades que un hecho exista dos veces en la naturaleza, no quiere decir, que se de así en la conciencia. Eso quiere decir que hay algo en la (materia, universo, espacio como le llamen) que hace que un solo hecho se de en la conciencia, lo que supone es que si las partículas tendrían una diferencia de entropía seria el mondo para discriminar una de las posibilidades, sucediendo que tengamos una sola realidad en la conciencia. El hecho que se de una u otra dependería de una suerte al azar en la que la entropía jugaría un gran papel.

\subsection{Sobre la materia}
Hemos dicho que la realidad es una mera invención con una entropía muy baja de una conciencia superior, pero por que experimentamos limites, resistencias. Si fuésemos un holograma, no debería haber estas restricciones en el holograma? una cosa es ser un holograma y otra ser lo proyectado por el holograma.

Lo que es proyectado por el holograma, no es mas que restricciones y limites matemáticos de la razón en las que una cosa puede acaecer o no. Es decir la conciencia (o la razón) crea la materia con principios matemáticos.


Lo que existe son solo las conciencias, las conciencias se podrían imaginar como unas esferas por dentro, vacías, en la que se proyecta el universo que vemos, en el universo que podemos captar a través de nuestros sentidos. somos conciencias que nos comunicamos a través de nosotros por el lenguaje y todos vemos la misma esfera. Seriamos conciencias múltiples (universos múltiples, copias o proyecciones del mismo) que nos comunicaríamos entre nosotras a cerca de lo proyectado que experimentamos.


\subsection{La imposibilidad del Dios cristiano}

En la concepción cristiana se da una moral basada en principios divinos de la proveniencia, es decir dios sabe lo que hace, cuando lo hace y porque lo hace, llevar una vida bajo los mandatos implica que cuando uno no vive bajo estos mandamientos un castigo divino es enviado, salen mal los negocios, caen desastres, es decir, no podría salir nada bien. lo que lleva a pensar que dios no existe, por que no hay castigos divinos en el mundo. Los políticos siguen robando y los criminales siguen delinquiendo.

Esta concepción se ve un dios que mueve los hilos de la moral, Dios hace posible milagros, habría que tener una fe inconmensurable para que Dios de un milagro, como se supone que si actuamos mal, tendríamos lo opuesto a esto, recibiríamos un castigo. Esto implica que existirían los castigos divinos, pero los castigos legales y morales no tendrían la razón de existir, ya que el castigo entre los individuos estaría en un segundo plano. Algunos podrán objetar que dios actúa a través de los castigos humanos, pero no es el caso ya que no todos los criminales purgan por sus crímenes.

Con eso se demuestra que dios no actúa en el mundo, lo cual si dios interviniera en el mundo se establecería cierto orden lo que es una contradicción con el axioma de que la vida es caótica. queda claro que esta es un hecho innegable de que dios no actúa en el mundo.

Si Dios no puede intervenir en el mundo no tiene sentido tener un Dios, es decir las oraciones serian flatus voices, meros ruidos al viento. Lo cual en un sentido utlitarista Dios no tendría razón de ser. Si Dios interviniera en el mundo el sabotearía todo acto criminal.

Suerte aquellos que tienen la no intervención de Dios, ya que pone pruebas difíciles en la vida; lo que me lleva a pensar que tiene algún tipo de trastorno mental ya que lastima a los hijos, ya por dar la lección que el es superior o por cualquier otra lección, cualquiera que sea el caso lo que hace por elección propia demostrando que en lugar de querer ser bueno, actúa de forma mala, haciéndole la vida difícil al humano.


El cristianismo se invento como una vivencia desde lo mas humilde, para que los pobres se identificaran; es decir si el 80 por ciento son pobres, hay que darles a la masa un ser resiliente y pobre que todo lo puede con fe. Es decir dar la otra mejilla, no robar, etc. Son cosas con las personas pobres se identificaran, los ricos podrían seguir robando sin que los pobres se molesten por que hay que poner la otra mejilla, eso hace que el dios cristiano sea un humilde y resiciliente ser que no se interesa por una vida de lujos.

en resumen no tiene sentido tener un Dios que no interviene en el mundo desde un punto utilitarista, con esto se demuestra la imposibilidad de un Dios cristiano.



\subsection{Sobre la muerte}

Hay vida después de la muerte? La respuesta es si, hay una vida después de la muerte por que la conciencia depende de la estructura del cerebro, por la organización de este como neuronas, son las organización de las neuronas que permiten tener conciencia. Una vez que el cerebro para de trabajar el estado cuántico de las neuronas empezarían a desaparecer, lo cual es una manera lenta de dejar de existir.

Es por eso que hay personas que experimentan una vida después de la muerte porque el funcionamiento del cerebro sigue activo por que se encuentra vivo, pero sin posibilidad de ejercer animación sobre el cuerpo, es decir hay conciencia pero no hay alma.

Una forma de vivir eternamente, es una vida en que se mantiene el cerebro con actividad.

\subsection{Sobre las matemáticas y el lenguaje}

Las matemáticas son un lenguaje de la razón, ellas se basan en una estructura mental que abstrae conceptos y forma nuevas relaciones. En la teoría de categorías, existe un mecanismo como abstracción que permite crear un concepto a partir de otro concepto mas general. Este es el mecanismo de como la razón funciona, es por este hecho, el de las abstracciones que se permite tener conocimientos matemáticos y conocimientos de cualquier lenguaje.

La abstracción es la maquina del entendimiento junto con la composición. De un concepto se abstrae el concepto abstracto y con la composición se establece una relación entre el concepto y su concepto abstracto y así sucesivamente, los numero son una construcción de una función matemática llamada el sucesor.




\subsection{Conclusion}
Como se menciono en la introducción la idea era llegar a la demostración de Dios por medio de un fisicalismo, como se demostró Dios esta sujeto a las leyes de la entropía debido a que es un estado creado a partir de microestados. También se negó la existencia del universo y se dejo como un holograma a partir de una conciencia superior. Se describió lo que es el tiempo y que la vida es un proceso al azar.

Ademas se presento el concepto de la metafísica la cual es la base del siguiente articulo, demostrar la existencia de Dios desde un punto de vista metafísico, a saber Dios es una conciencia superior.







































































\end{document}

